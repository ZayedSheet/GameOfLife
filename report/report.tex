\documentclass[12pt]{article}

\usepackage{graphicx}
\usepackage{paralist}
\usepackage{amsfonts}
\usepackage{amsmath}
\usepackage{hhline}
\usepackage{booktabs}
\usepackage{multirow}
\usepackage{multicol}
\usepackage{url}

\oddsidemargin -10mm
\evensidemargin -10mm
\textwidth 160mm
\textheight 200mm
\renewcommand\baselinestretch{1.0}

\pagestyle {plain}
\pagenumbering{arabic}

\newcounter{stepnum}

%% Comments

\usepackage{color}

\newif\ifcomments\commentstrue

\ifcomments
\newcommand{\authornote}[3]{\textcolor{#1}{[#3 ---#2]}}
\newcommand{\todo}[1]{\textcolor{red}{[TODO: #1]}}
\else
\newcommand{\authornote}[3]{}
\newcommand{\todo}[1]{}
\fi

\newcommand{\wss}[1]{\authornote{blue}{SS}{#1}}

\title{Assignment 4, Part 1, Specification}
\author{SFWR ENG 2AA4}

\begin {document}

\maketitle
This Module Interface Specification (MIS) document contains modules, types and
methods for implementing the state of a game of Conway's Game of Life as well as viewing
the state. Game of Life is a grid of square cells where each is in one of two possible states,
alive or dead. Each cell interacts with its 8 surrounding neighbours to determine the next state.
The Game's rules are as follows:

\begin{enumerate}
\item Any live cell with fewer than two live neighbours dies, as if by underpopulation.
\item Any live cell with two or three live neighbours lives on to the next generation.
\item Any live cell with more than three live neighbours dies, as if by overpopulation.
\item Any dead cell with exactly three live neighbours becomes a live cell, as if by reproduction.
\end{enumerate}



In applying the specification, there will be cases that involve undefinedness.
We will interpret undefinedness following~\cite{Farmer2004}:

If $p: \alpha_1 \times .... \times \alpha_n \rightarrow \mathbb{B}$ and any of
$a_1, ..., a_n$ is undefined, then $p(a_1, ..., a_n)$ is False.  For instance,
if $p(x) = 1/x < 1$, then $p(0) = \text{False}$.  In the language of our
specification, if evaluating an expression generates an exception, then the
value of the expression is undefined.

\bibliographystyle{plain}
\bibliography{SmithCollectedRefs}

\newpage

\section* {Cell Module}

\subsection*{Module}

Cell

\subsection* {Uses}

N/A

\subsection* {Syntax}

\subsubsection* {Exported Constants}

None

\subsubsection* {Exported Types}

CellT = \{Alive, Dead\}

\subsubsection* {Exported Access Programs}

None

\subsection* {Semantics}

\subsubsection* {State Variables}

None

\subsubsection* {State Invariant}

None

\newpage


\section* {Game State ADT Module}

\subsection* {Template Module}

StateT

\subsection* {Uses}

N/A

\subsection* {Syntax}


\subsubsection* {Exported Access Programs}

\begin{tabular}{| l | l | l | p{5cm} |}
\hline
\textbf{Routine name} & \textbf{In} & \textbf{Out} & \textbf{Exceptions}\\
\hline
new StateT & $M$ & StateT & invalid\_argument\\
\hline
updateState &  &  & none\\
\hline
getCell & $\mathbb{N}$,$\mathbb{N}$ & CellT & invalid\_argument\\
\hline
size & & $\mathbb{N}$ & \\
\hline
\end{tabular}

\subsection* {Semantics}

\subsubsection* {State Variables}

$S$: Matrix \#$Board Grid$

\subsubsection* {State Invariant}

$\forall e \in M : (|M| = |e|) $ \#size of all elements in M are equal to size of M ensuring square matrix

\subsubsection* {Assumptions \& Design Decisions}

\begin{itemize}

\item The StateT constructor is called before any other access
  routine is called on that instance. Once a StateT has been created, the
  constructor will not be called on it again.

\item The constructor can be used with an empty matrix, however the user should
 understand no data can be stored or received using this.

\item Each element must either be Dead or Alive. (The initializer automatically sets all
 elements to dead to begin with)

\item For better scalability, this module is specified as an Abstract Data Type
  (ADT) instead of an Abstract Object. This would allow multiple games to be
  created and tracked at once by a client.

\item Any area outside the gameboard will be considered Dead Cells and will not be affected
by the game in any way. For example if the board is 20x20 then cell 19,20 will be considered
dead and wont be affected by cell 19,19.

\item Getter functions are provided, though violating the property of being
  essential, to give a would-be view function easy access to the state of the
  game, as well as to make testing easier.

\end{itemize}

\subsubsection* {Access Routine Semantics}

new StateT($M$):
\begin{itemize}
\item transition: $S := M$

\item output: $\mathit{out} := \mathit{self}$
\item exception: $exc := (!(\forall e \in M : |M| = |e|) \Rightarrow \text{invalid\_argument})$ \#ensures square matrix
\end{itemize}

\noindent updateState():
\begin{itemize}
  \item transition: $S$ := M such that $(\forall i,j|i,j \in [0..|S|-1]: M[i][j] = \text{updateCell}(S,i,j))$
\item exception: none
\end{itemize}

\noindent getCell(i,j):
\begin{itemize}
\item output: $out := S[i][j]$

\item exception: $exc := (!(\text{validPoint}(i,j)) \Rightarrow \text{invalid\_argument})$

\end{itemize}

\noindent size():
\begin{itemize}
\item output: $\mathit{out} := |S|$
\item exception: None
\end{itemize}

\subsection*{Local Types}

Matrix = seq of (seq of CellT)

\subsection*{Local Functions}
updateCell : Matrix x $\mathbb{N}$ x $\mathbb{N} \rightarrow \mathbb{B}$\\
updateCell(m, i, j) $\equiv$

\begin{tabular}{|p{4cm}|p{5cm}|l|}
\hhline{|-|-|-|}
$S[i][j] = Alive$ & countLiveCells($i$,$j$) = 2 & $m[i][j] := Alive$\\
\hhline{|~|-|-|}
 & countLiveCells($i$,$j$) = 3 & $m[i][j] := Alive$\\
\hhline{|-|-|-|}
$S[i][j] = Dead$ & countLiveCells($i$,$j$) = 3 & $m[i][j] := Alive$\\
\hhline{|-|-|-|}
\end{tabular}\\\\\\
countLiveCells: $\mathbb{N}$ x $\mathbb{N} \rightarrow \mathbb{N}$\\
countLiveCells(i,j)$\equiv$ $+(x,y:\mathbb{Z}| x \in \{i-1,i+1,i\} \land y \in \{j-1,j+1,j\} \land \text{validPoint}(i,j) \land !(i = x \land j = y): 1)$\\\\
validPoint: $\mathbb{Z}$ x $\mathbb{Z} \rightarrow \mathbb{B}$\\
validPoint(i,j) $i \leq 0 \land j \leq$ 0 \land $\equiv$ $i < |S| \land j < |S| $ \#Used integer rather than nat for params because i-1 can be passed in when running countLiveCells

\newpage

\section* {View Module}

\subsection* {View Module}

View

\subsection* {Uses}

GameState

\subsection* {Syntax}


\subsubsection* {Exported Access Programs}

\begin{tabular}{| l | l | l | p{5cm} |}
\hline
\textbf{Routine name} & \textbf{In} & \textbf{Out} & \textbf{Exceptions}\\
\hline
initializeBoard & s: string & StateT & runtime\_error\\
\hline
writeBoard & StateT, s: string &  & \\
\hline
printState& StateT & &  \\
\hline
\end{tabular}

\subsection* {Semantics}

\subsubsection* {Environmental Variables}
input: File representing initial board state
output: File representing output board state

\subsubsection* {State Variables}

None

\subsubsection* {State Invariant}

None

\subsubsection* {Assumptions \& Design Decisions}

\begin{itemize}
\item Users will follow the input file format correctly

\end{itemize}

\subsubsection* {Access Routine Semantics}

\noindent initialize($s$)
\begin{itemize}
\item transition: read data from the file input.txt associated with the string s.
  Use this data to initialize an instance of the GameState module. This function will
  first use the input file to create a matrix with the appropriate cells. It will use this
  matrix to initialize a GameState.

  The text file has the following format, where all data values ina row are two numbers
  seperated by only a sinle comma, except for the first row which is a single number representing the
  size of the matrix. For example, if the first line is 20, the matrix will be 20 by 20.
   This is assumed to be an appropriate number. The following lines represent
  the locations where a cell is to be Alive. $row_n$ and $col_n$ are numbers representing
  the co-ordinates of an Alive cells where $x \geq 0$. Cells are automatically instantiated to be Dead
  otherwise. For example, $row_0, col_0$ means that Matrix[$row_0$,$col_0$] is Alive.

  \begin{equation}
    \begin{array}{ccccccc}
      Number\\
      row_1, & col_1\\
      row_2, & col_2\\
      row_3, & col_3
      \\
      ..., & ...
      \\
      row_n, & col_n \\
    \end{array}
  \end{equation}

\item exception: if the file $s$ is not found, throw invalid\_argument. If any integer value
  $row_n$ or $col_n$ (where $n \geq 0$) is greater than or equal to $Number$ (first row), throw invalid\_argument.
\end{itemize}


\noindent writeBoard ($b$,$s$)
\begin{itemize}
\item transition: First, b.size()  is written to a file (this being the size of the board).
  Then, for all integer i values less than b.size(), and for all integer j values less than
  b.size(), if b.getCell($i,j$) = Alive, write to the file a new line character followed by the integer value i followed by a
  comma followed by integer value j. The output file should
  have the format seen below, where Number represents a single integer number (size of board),
  and the following lines represent the row number followed by a comma followed by the column
  number where a cell is Alive in b.

  \begin{equation}
    \begin{array}{ccccccc}
      Number\\
      row_1, & col_1\\
      row_2, & col_2\\
      row_3, & col_3
      \\
      ..., & ...
      \\
      row_n, & col_n \\
    \end{array}
  \end{equation}

\item exception: none
\end{itemize}


\noindent printState($b$):
\begin{itemize}
\item transition: this function initializes a string variable.
  for all integer i values less than b.size() (for all integer j values less than
  b.size(), if b.getCell($i,j$) = Dead, it appends to the string [ ], otherwise it appends [o])
  it appends a new line character at the end of each row (when i is about to change).
  It then prints this string. For example, if b = [[Dead, Dead, Alive], [Dead, Alive, Dead], [Dead, Dead, Alive]]
  the string will be "[ ][ ][o]n[ ][o][ ]n[ ][ ][o]" where n represents a new line character.


\item exception: None
\end{itemize}

\newpage
  \section* {Critique of Design}
  \\
  First and foremost, I believe my design could focus a little more on modularity.
  Because my modules do not have high cohesion and low coupling because my printString method in
  my View module highly depends on my GameState method. I believe it would make more sense to add a
  toString() method in gameState that returns the gameState as a string. This would result in
  higher cohesion and lower coupling. \\\\

  My design also violates essentiality due to the additional getter methods for
  my GameState function that I used to help with unit testing. This can be solved by simply removing
  those functions.\\\\

  My design could furthur improve generality and separation of concerns by including
  a generic matrix class.\\\\

  In terms of information hiding my program does this well by using state variables and private functions,
  however I do have a getCell getter function to help for testing and writing. I decided to use
  a getCell rather than a getState (which would return the matrix) to further include information hiding.
  A matrix would output all information rather than one cell and therefore a getCell function
  further emphasizes information hiding.\\\\

  In terms of minimality, I believe my GameState functions highly incorporate this however
  while I believe my read and write functions incorporate this it might not be to as high an
  extent because reading to a file and writing to a file can be more complicated than other
  simple tasks.\\\\

  Lastly, in terms of consistency I believe my design performs well with respect to this quality
  because input parameters, function names have consistent writing conventions, and the logic
  throughout my code is also very consistent.\\\\



\end {document}